\chapter{Conclusions}
In this laboratory work we were able to prepare a \emph{tropos} chiral compound to use as chiral ligand for rhodium(I) in the addition of aryl\-boronic acids to 1-nitro-alkenes.

The strategy followed by our group for the homocoupling of 2-iodo-1-methoxy\-naphthalene was not successful because of a side reaction that can occur on methoxy\-naphthalene and not on methoxy\-benzenes, due to the presence of the $\alpha$' hydrogen.
As an alternative, using the Ullmann reaction, coupling was performed (by other groups) without problems and was possible to proceed with the subsequent steps to obtain the \emph{tropos} ligand \cmpd+{legante}.

The crucial step of the synthesis is reaction with \ce{PCl3} first with binaphthyl \cmpd+{dhn} and then with protected deoxychilic acid \cmpd{amda}. In this step working under inert atmosphere is a strict condition to avoid hydrolysis of \ce{PCl3}. %and derivatives that are not reactive towards our compounds. 
We obtained only 5~\% yield, so we can suppose that some moisture entered the reaction apparatus, probably during filtration of binaphthyl chloro\-phosphite and the transfer of this to the next reaction with \cmpd+{amda}.

Performing previous steps afforded pure compounds in modest to good yields after flash chromatography.

In the target enantio\-selective reaction we observed a poor conversion (17~\%) and the formation of 4 products: \emph{cis} (1\rS,2\rS), (1\Rs,2\Rs) and \emph{trans} (1\rS,2\Rs), (1\Rs,2\rS) [(1{\slshape ?},2{\slshape ?})-2-nitro-cyclo\-hexyl]benzene \cmpd+{ncb}. \emph{Trans} product was prevalent having a diastereo\-meric excess of 25~\%. The enantio\-meric excess of the \emph{trans} product was 60~\%.